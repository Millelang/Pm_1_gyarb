\documentclass[11p]{article}
% Packages
\usepackage{amsmath}
\usepackage{graphicx}
\usepackage{fancyheadings}
\usepackage[swedish]{babel}
\usepackage[
    backend=biber,
    style=authoryear-ibid,
    sorting=ynt
]{biblatex}
\usepackage[utf8]{inputenc}
\usepackage[T1]{fontenc}
%Källor
\addbibresource{references.bib}
\graphicspath{ {./images/} }

% Lite variabler
\def\email{Milton.langstrom@ga.ntig.se}
\def\foottitle{Pm 1}
\def\name{Milton Långström}

\title{PMmall \\ \small Gymnasiearbete}
\author{\name}
\date{\today}

\begin{document}

% fixar sidfot
\lfoot{\footnotesize{\name \\ \email}}
\rfoot{\footnotesize{\today}}
\lhead{\sc\footnotesize\foottitle}
\rhead{\nouppercase{\sc\footnotesize\leftmark}}
\pagestyle{fancy}
\renewcommand{\headrulewidth}{0.2pt}
\renewcommand{\footrulewidth}{0.2pt}

% i Sverige har vi normalt inget indrag vid nytt stycke
\setlength{\parindent}{0pt}
% men däremot lite mellanrum
\setlength{\parskip}{10pt}

\maketitle

\section{Bakgrund}

\subsection{Vad är tillgänglighet?}
Tillgänglighet innebär kortfattat att alla miljöer och tjänster ska vara tillgängligt för alla och är även en rättighet. Detta kan innebära att bussarnas ingångar är så pass låga så att en rullstolsburen person kan ta sig på bussen eller att vårdcentraler har en blindskrift version av deras skyltar.

\subsection{Vad är speltillgänglighet?}
Speltillgänglighet innebär att ta bort begränsningar på spel som gör så att folk som har någon typ av funktionsnedsättning kan köra spelen utan problem, tex färgblindhetsfilter eller att det går att ändra textstorlek i spelet. Detta är viktigt eftersom att minst en tredjedel utav världens gamers har en funktionsnedsättning \parencite{KineticJournals} vilket gör speltillgänglighet till en nödvändighet inom spel, så att det ska vara tillgängligt för alla.


\n

\subsection{Vad är Game accessibility guidelines?}
Game accessibility guidelines är en hemsida som konstant updateras och innehåller ett flertal speltillgänglighetregler som spelutvecklare kan följa för att göra sitt spel tillgängligt för personer med funktionsnedsättningar. Dessa regler innehåller allt ifrån att det ska gå att ändra kontrollerna i ett spel till att gömma alla föremål som inte går att interagera med. \parencite{Gameaccessibilityguidelines}

\n

\subsection{Vad är ett battle royale?}
Att ett spel är ett "battle royal" innebär oftast att ett visst antal spelare spelar på en och samma karta med målet att vara den sista som överlever, vilket till att den sista överlevaren vinner.

\subsection{Vad är fortnite?}
Fortnite är ett third-person shooter battle royale spel där hundra stycken spelare hoppar nerifrån en flygande buss till en massiv värld med flera olika namngedda platser. \parencite{Fornite}

\subsection{Vad är Apex legends?}
Apex legends är ett first-person shooter battle royale spel gjort utav Electronic arts med över 20 olika karaktärer som alla har sina unika egenskaper och krafter, Apex legends går ut på att 100 spelare landar på en karta och samlar på sig futuriska vapen med syftet att döda alla andra spelare och vara den sista överlevaren medans en storm omringar kartan och minskar spelområdet.     \parencite{Apex}

\subsection{Vad är Warzone 2.0?}
Warzone 2.0 är ett firstperson shooter battle royale där 100-200 spelare bereonde på vilket spelläge man spelar hoppar falskräm från ett plan ner till en karta som efterliknar Iraks huvudstad Bagdad. Sedan efter alla spelare har landat så har man ett enda uppdrag, vilket är att eliminera alla andra spelare med hjälp utav de vapen som finns utspridda på kartan. Under spelets gång så omringas kartan med giftig gas vilket leder till att spelområder krymper vilket tvingar alla spelare närmare varandra, ända tills det endast är en spelare kvar \parencite{Warzone2.0}



\printbibliography

\end{document}
